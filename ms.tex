%% This is emulateapj reformatting of the AASTEX sample document
%%
\documentclass[preprint2]{aastex}
%\documentclass[iop]{emulateapj}

\usepackage{natbib}
\usepackage{amssymb,amsmath}
\usepackage[backref,breaklinks,colorlinks,citecolor=blue]{hyperref}
\usepackage[all]{hypcap}
\renewcommand*{\backref}[1]{[#1]}

\usepackage{acronym}


\bibliographystyle{apj}

\newcommand{\vdag}{(v)^\dagger}
\newcommand{\myemail}{tdm@astro.princeton.edu}

\newcommand{\ntotal}{8826}
\newcommand{\nfail}{1584}
\newcommand{\ncalc}{7242}
\newcommand{\nval}{1500}
\newcommand{\nvalnew}{1000}
\newcommand{\nfp}{500}
\newcommand{\nfpnew}{200}
\newcommand{\kepler}{\textit{Kepler}}
\newcommand{\vespa}{\texttt{vespa}}
\newcommand{\isochrones}{\texttt{isochrones}}
\newcommand{\bvec}[1]{{\ensuremath{\boldsymbol{#1}}}}

\defcitealias{Morton:2012}{M12}

\acrodef{fpp}[FPP]{false positive probability}
\acrodef{koi}[KOI]{\kepler\ Object of Interest}
\acrodefplural{koi}[KOIs]{\kepler\ Objects of Interest}
\acrodef{nexsci}[NExScI]{NASA Exoplanet Science Institute}
\acrodef{mcmc}[MCMC]{Markov Chain Monte Carlo}
\acrodef{trilegal}[TRILEGAL]{TRIdimensional modeL of thE GALaxy}

%% You can insert a short comment on the title page using the command below.

\slugcomment{}

\shorttitle{False Positive Probabilities for KOIs}
\shortauthors{Morton et al.}

%% This is the end of the preamble.  Indicate the beginning of the
%% paper itself with \begin{document}.

\begin{document}

%% LaTeX will automatically break titles if they run longer than
%% one line. However, you may use \\ to force a line break if
%% you desire.

\title{False Positive Probabilities for all Kepler Objects of Interest: \\
        \nvalnew\ newly validated planets and \nfpnew\ likely false positives}

%% Use \author, \affil, and the \and command to format
%% author and affiliation information.
%% Note that \email has replaced the old \authoremail command
%% from AASTeX v4.0. You can use \email to mark an email address
%% anywhere in the paper, not just in the front matter.
%% As in the title, use \\ to force line breaks.

\author{Timothy D. Morton}
\affil{Department of Astrophysical Sciences, Princeton University}

%\altaffiltext{1}{Department of Astrophysics, Princeton University}

%% Mark off your abstract in the ``abstract'' environment. In the manuscript
%% style, abstract will output a Received/Accepted line after the
%% title and affiliation information. No date will appear since the author
%% does not have this information. The dates will be filled in by the
%% editorial office after submission.

\begin{abstract}
We present the results of applying a fully automated transit signal
false positive probability calculating procedure to every
\kepler\ Object of Interest (KOI) from the Q1-Q16 catalog release.
Out of \ncalc\ KOIs, we determine that \nval\ have probabilities
$<$1\% to be astrophysical false positives, and thus may be considered
validated planets.  \nvalnew\ of these have not yet been validated or
confirmed by other methods.  In addition, we identify \nfp\ KOIs
likely to be false positives ($>$90\% probability), \nfpnew\ of which
have not yet been identified as such. A side product of these
calculations is full stellar property posterior samplings for every
host star, modeled as single, binary, and triple.  These calculations
use \vespa, a publicly available Python package able to be easily
applied to any transiting exoplanet candidate.
\end{abstract}

%% Keywords should appear after the \end{abstract} command. The uncommented
%% example has been keyed in ApJ style. See the instructions to authors
%% for the journal to which you are submitting your paper to determine
%% what keyword punctuation is appropriate.

%% Authors who wish to have the most important objects in their paper
%% linked in the electronic edition to a data center may do so in the
%% subject header.  Objects should be in the appropriate "individual"
%% headers (e.g. quasars: individual, stars: individual, etc.) with the
%% additional provision that the total number of headers, including each
%% individual object, not exceed six.  The \objectname{} macro, and its
%% alias \object{}, is used to mark each object.  The macro takes the object
%% name as its primary argument.  This name will appear in the paper
%% and serve as the link's anchor in the electronic edition if the name
%% is recognized by the data centers.  The macro also takes an optional
%% argument in parentheses in cases where the data center identification
%% differs from what is to be printed in the paper.

\keywords{}


\section{Introduction}

The \kepler\ mission has revolutionized our understanding of
exoplanets.  Among many other important discoveries, \kepler\ has
identified several previously unsuspected features of planetary
systems, such as the prevalence of planets between the size of Earth
and Neptune, and a population of very compact multiple-planet
systems. And perhaps most notably, it has enabled for the first time
estimates of the occurrence rates of small planets ($\gtrsim$1
$R_\oplus$) out to orbits of about one year.  It is important to
remember, however, that these revolutionary discoveries depend
intimately on another revolution---how to interpret transiting planet
\textit{candidate} signals in the absence of unambiguous positive
confirmation of their veracity.

Before \kepler, every survey searching for transiting exoplanets had a 

\ac{fpp} and then \ac{fpp}.

%%%%%%%%%%%%%%%%%%%%%%%%%%%%%%%%%%%%%%%%%%%%%%%%%%%%%%

\section{Methods}
\label{sec:methods}

In this work, we apply the fully automated \ac{fpp}-computing
procedure described in \citet[][hereafter
  \citetalias{Morton:2012}]{Morton:2012} to \ncalc\ \acp{koi} (see
\autoref{sec:data} for details).  While we refer the reader to
\citetalias{Morton:2012} for a detailed description of the method, we
outline it briefly in this section.  The procedure as implemented here
is now publicly available in the Python module
\vespa\footnote{\url{https://github.com/timothydmorton/vespa}}
\citep{vespa}.

%%%%%%%%%%%%%%%%%%%%%%%%%%%%%%%

\subsection{False Positive Probabilities}
\label{sec:methods:fpp}

The basic idea of \vespa\ is to assign probabilities to different
astrophysical hypotheses that might describe a transiting planet
candidate signal.  If $\{H_i\}$ is the set of all considered
hypotheses, the probability for any given model $i$ is
\begin{equation}
  \label{eq:prob}
  \mathrm{Pr}\left(H_i\right) = \frac{\pi_i \mathcal
    L_i}{\displaystyle \sum_j \pi_j \mathcal L_j},
\end{equation}
where $\pi_i$ is the ``hypothesis prior'' and $\mathcal L_i$ is the
``hypothesis likelihood''\footnote{This factor is more widely known as
  the ``Bayesian evidence'' or ``marginalized likelihood'';
  \citet{Morton:2014b} argues for the term ``hypothesis likelihood,''
  as it can be clarifying to think of it that way.}
The prior represents how intrinsically probable the hypothesized
scenario is to exist, and the likelihood represents how closely the
shape of the observed transit signal matches with the expected shape
of a signal produced by the hypothesis.

\vespa\ models an eclipse signal as a simple trapezoid, parametrized
by depth $\delta$, total duration $T$, and shape parameter $T / \tau$,
where $\tau$ is the ``ingress/egress'' duration (such that a
completely V-shaped transit has $T/\tau = 2$).  For the transit signal
being evaluated, the joint posterior probability density function
(PDF) of these shape parameters is sampled with \ac{mcmc}, using the
\texttt{emcee} sampler \citep{emcee}.  This allows the likelihood for
each hypothesis to be determined by simulating a physically realistic
population of the hypothesized astrophysical scenario and using this
population to define the PDF for the trapezoidal parameters under the
hypothesis.  The likelihood is then
\begin{equation}
  \label{eq:lhood}
  \mathcal L_i = \displaystyle \int p_\mathrm{sig}\left(\bvec{\theta}\right)
                                    p_i\left(\bvec{\theta}\right)\,d\bvec{\theta},
\end{equation}
where $\bvec{\theta}$ is the vector of trapezoidal shape parameters,
$p_\mathrm{sig}$ is the posterior PDF of the signal, and $p_i$ is the
PDF for the parameters under hypothesis $i$.\footnote{$\mathcal L_i$
  may be seen to be the ``evidence'' or ``marginalized likelihood'' of
  the trapezoidal model under hypothesis $i$, with $p_\mathrm{sig}$
  being the likelihood and $p_i$ being the prior, integrated over the
  $\bvec{\theta}$ parameter space.  But for clarity, and for
  continuity with previous publication, we continue to call $\mathcal
  L_i$ the ``likelihood'' for hypothesis $i$.}

Observational constraints are incorporated in two different ways.
First, photometric (or spectroscopic/asteroseismic) measurements of
the target star are folded into the population simulations of each
hypothesis (see \autoref{sec:methods:stellar}).  All other constraints are
applied to narrow down which simulated instances of each scenario may
be counted in the final prior and likelihood evalulations; for example,
only blended eclipsing binaries with secondary eclipse depths
shallower than the observed limits contribute to the construction of
the $p_i$ trapezoidal shape parameter PDF.

The steps \vespa\ takes to calculate the \ac{fpp} of a transit signal
are thus as follows:
\begin{enumerate}
\item Fit trapezoid model to observed transit signal using \ac{mcmc}.
\item Generate population simulations for each hypothesis scenario
  being considered (conditioned on available observations of the
  target star; see \autoref{sec:methods:stellar}).
\item Fit each simulated eclipse in each scenario with a trapezoid
  model (using least-squares optimization).
\item Evaluate priors and likelihoods for each hypothesis, taking into
  account all available observational constraints, in order to
  calculate the \ac{fpp}.
\end{enumerate}

%%%%%%%%%%%%%%%%%%%%%%

\subsection{Stellar Properties}
\label{sec:methods:stellar}

The most substantial difference between the current implementation of
\vespa\ and the procedure documented in \citetalias{Morton:2012} is
how stellar properties are treated.  Previously, either the target
star's mass and radius were explicitly provided, or they were randomly
generated according to the stellar population expected along the line
of sight by the \ac{trilegal} Galactic stellar popultion synthesis
tool, but constrained to agree with some observed color(s) of the star
(e.g. $J-K$), to within some specified tolerance.  This strategy was
used both to generate the host stars for the transiting planet model
and the binary and triple stars for the EB and HEB false positive
models.

The new method now used by \vespa\ uses the \isochrones\ Python module
\citep{isochrones} to fold in observational constraints on the host
star.  At its core, \isochrones\ performs 3-D linear interpolation in
mass--[Fe/H]--age parameter space for a given stellar model grid.
This method of stellar modeling for FPP calculation debuted in
\citet{Montet:2015} and is explained there in more detail.  Instead of
randomly generating stars (or binary or triple systems of stars) from
a predefined distribution and culling them to agree with observed
colors, \emph{all} available constraints on the target star are used
to condition a direct fit of either a single--, binary--, or
triple--star model to the Dartmouth grid of stellar models.  This fit
is done using \texttt{MultiNest} \citep{Feroz:2009, Feroz:2011,
  Feroz:2013}, via the \texttt{PyMultiNest} wrapper
\citep{Buchner:2014}.  Monte Carlo samples of stellar properties for
the population simulations are then drawn directly from these
posterior samples.

As a result, \vespa\ creates full posterior samplings of the physical
properties of the host star, modeled as a single, binary, and triple
star system, as a by-product of the \ac{fpp} calculation.  Parameters
directly fitted for in this process are stellar mass, age, [Fe/H],
$A_V$ extinction, and distance.  For binary and triple fits, secondary
and/or tertiary mass parameters are added, with all other parameters
assumed to be the same among all components.  Photometric observations
upon which these fits are conditioned are assumed to be the sum of all
components.  If spectroscopic and/or asteroseismic measurements are
used (e.g., constraints on effective temperature or stellar surface
gravity), they are assumed to relate to only the primary star.  Priors
used in these fits are listed in \autoref{table:priors}.  Posterior
chains of all other stellar parameters of interest (e.g., temperature,
surface gravity, radius, etc.) are derived from the chains of fitting
parameters by evaluting the stellar models using \isochrones.

\capstartfalse
\begin{deluxetable}{cc}
\tablewidth{0pt}
\tabletypesize{\scriptsize}
\tablecaption{Priors used in stellar property fits}
\tablehead{
\colhead{Parameter} &
\colhead{Prior}}
\startdata
Primary mass $M_A$ & $\propto M_A^{-2.35},~M_A > 0.1$ \\
Secondary mass $M_B$ & $\propto (M_B/M_A)^{0.3},~0.1 <= M_B < M_A$ \\
Tertiary mass $M_C$ & $\propto (M_C/M_A)^{0.3},~0.1 <= M_C < M_B$ \\
Age {[}Gyr{]} & $\mathcal U(1,15)$ \tablenotemark{a}\\
{[}Fe/H{]} & $\frac{0.8}{0.15} \mathcal N(0.016, 0.15) +
\frac{0.2}{0.22} \mathcal N(-0.15, 0.22)$  \tablenotemark{b}\\
$A_V$ {[}mag{]} & $\mathcal U(0, A_{V, \mathrm{max}})$
\tablenotemark{c} \\
Distance $d$ & $\propto d^2$ 
\enddata
\tablenotetext{a}{The age range for the Dartmouth stellar model grids
  used.}
\tablenotetext{b}{Double-Gaussian fit to measured local stellar metallicity
  distribution \citep{Hayden:2015, Casagrande:2011}.}
\tablenotetext{3}{Maximum allowed value is the Galactic extinction at
  infinity calculated along the star's line of sight, according to \citet{Schlegel:1998}.}
\label{table:priors}
\end{deluxetable}
\capstarttrue

%%%%%%%%%%%%%%%%%%%%%%%%%%%%%%%%%%%%%%%%%%%%%%%%%%%%%%

\section{Data and Constraints}
\label{sec:data}

The goal of this work is to calculate the \ac{fpp} for every \ac{koi},
regardless of classification as CONFIRMED, CANDIDATE, or
FALSE POSITIVE.  As such, we begin with a list of
\ntotal\ \acp{koi} from the cumulative table at the \ac{nexsci}
Exoplanet Archive.  However, for various reasons, the ancillary data
and constraints that we describe below are not available for every
\ac{koi}, and so in the end we present \ac{fpp} results for
\ncalc\ \acp{koi}.


%%%%%%%%%%%%%%%%%%%%%%%%%%%%%%%%%%%%%%%%%%%%%%%%%%%%%%

\section{Results}
\label{sec:results}

%%%%%%%%%%%%%%%%%%%%%%%%%%%%%%%

\subsection{Stellar Properties}
\label{sec:results:stars}

%%%%%%%%%%%%%%%%%%%%%%%%%%%%%%%

\subsection{False Positive Probabilities}
\label{sec:results:fpp}

%%%%%%%%%%%%%%%%%%%%%%%%%%%%%%%%%%%%%%%%%%%%%%%%%%%%%%

\section{Conclusions}
\label{sec:conclusions}

\acknowledgments
It is a pleasure to thank
\ldots\

\clearpage
\bibliography{ms}
\clearpage


\end{document}

%%
%% End of file `sample.tex'.
